\documentclass[11pt,a4paper]{article}

%%%%%%%%%%%%%%%%%%%%%%%%%%

\usepackage{latexsym}
\usepackage{amssymb}
\usepackage{amsmath}
\usepackage{amsfonts}
\usepackage{color}
\usepackage{graphicx}


%%%%%%%%%%%%%%%%%%%%%%%%%%

\setlength{\textheight}{25cm}
\setlength{\textwidth}{17cm}
\setlength{\topmargin}{-3cm}
\setlength{\oddsidemargin}{-5mm}
\setlength{\evensidemargin}{0.5cm}


%%%%%%%%%%%%%%%%%%%%%%%%%%

\newcommand{\red}{\textcolor{red}}
\newcommand{\blue}{\textcolor{blue}}
\newcommand{\white}{\textcolor{white}}

%%%%%%%%%%%%%%%%%%%%%%%%%%
 
\parindent=0mm %indentado de comienzo de p\'arrafo
\parskip=0mm %separaci\'on entre p\'arrafos

\thispagestyle{empty} %paginaci\'on
\date{ }

%%%%%%%%%%%%%%%%%%%%%%%%%%
%%%%%%%%%%%%%%%%%%%%%%%%%%%%%%%%%%%%%%%%%%%%%%%%%

\newcounter{ex}
\newcommand{\ejer}{\stepcounter{ex}\textbf{\theex.-}\ }
\newcommand{\sen}{\sin}

%%%%%%%%%%%%%%%%%%%%%%


\begin{document}

\hrule \vskip 1mm

\noindent {\large \sc C\'alculo 2}:   \quad $1^0$ del grado de  Matem\'aticas y doble grado MAT-IngINF \hfill  2022/23


\vskip 1mm

{\bf Ejercicio 5} (entrega el 16.05.23) \hskip 5mm {\bf APELLIDOS, Nombre}: TARRASA MARTÍN, Alberto
\vskip 1mm \hrule

\vskip 1mm

\hfill {\footnotesize \it  (Para la respuesta usa solo la cara de una p\'agina)}

\vskip 3mm

\ejer  El potencial debido a una carga $Q$ en $(0,0,0)$ viene dado por
$\displaystyle \Phi(x,y,z)=\frac{1}{4\pi} \frac{Q}{\sqrt{x^2+y^2+z^2}}$
y el campo el\'ectrico correspondiente es ${\bold E}=-\nabla \Phi$. 
\begin{enumerate}
\item Sea $\mathcal{S}_R$ la esfera de radio $R$ centrada  en el origen. Encuentra el flujo el\'ectrico del campo  ${\bold E}$ hacia el exterior de la esfera $\mathcal{S}_R$ y comprueba que solo depende de la constante $Q$ pero no del radio $R$.
\item Calcula la divergencia de ${\bold E}$ y explica por qu\'e no se puede usar el Teorema de Gauss en este caso.
\end{enumerate}

\vskip 5mm

{\bf SOL.:}\\
1. Para empezar, calculamos la expresión del campo eléctrico:
\[
\bold E(x,y,z)=-\nabla\Phi=\left(\dfrac{\partial\Phi}{\partial x},\dfrac{\partial\Phi}{\partial y},\dfrac{\partial\Phi}{\partial z}\right)=
\dfrac{Q}{4\pi}\left(\dfrac{x}{\sqrt{(x^2+y^2+z^2)^3}}, \dfrac{y}{\sqrt{(x^2+y^2+z^2)^3}}, \dfrac{z}{\sqrt{(x^2+y^2+z^2)^3}} \right)
\]

A continuación, parametrizamos la esfera $x^2+y^2+z^2=R^2$ con coordenadas esféricas:
\[
\mathcal S_R=\gamma(\theta,\varphi)=(R\sen\theta\cos\varphi,R\sen\theta\sen\varphi,R\cos\theta), \text{ con } (\theta,\varphi)\in[0,\pi]\times[0,2\pi]=D \text{ y } R>0
\]
Por último, calculamos $T_\theta$ y $T_\varphi$:
\[
T_\theta=\dfrac{\partial\gamma}{\partial\theta}=(R\cos\theta\cos\varphi,R\cos\theta\sen\varphi,-R\sen\theta), \quad T_\varphi=\dfrac{\partial\gamma}{\partial\varphi}=(-R\sen\theta\sen\varphi,R\sen\theta\cos\varphi,0)
\]
De forma que $T_\theta \times T_\varphi = (-R^2\sen^2\theta\cos\varphi, R^2\sen^2\theta\sen\varphi, R^2\sen\theta\cos\theta)$.\\
Así, el flujo eléctrico del campo eléctrico {\bf E} hacia el exterior de la esfera $\mathcal S_R$ es:
\[\displaystyle
\iint_\mathcal{S} {\bf E}\cdot dA=\iint_D {\bf E}(\gamma(\theta,\varphi))\cdot (T_\theta\times T_\varphi)\ d\theta d\varphi=
\]
\[=\iint_D \dfrac{Q}{4\pi R^2}(\sen\theta\cos\varphi,\sen\theta\sen\varphi,\cos\theta)\cdot R^2(-\sen^2\theta\cos\varphi,\sen^2\theta\sen\varphi,\sen\theta\cos\theta)\ d\theta d\varphi=
\]
\[
=\dfrac{Q}{4\pi}\int^{2\pi}_0\left( \int^{\pi}_0\sen\theta\ d\theta\right)\ d\varphi=\dfrac{Q}{4\pi}[-\cos\theta]^{\pi}_0[\varphi]^{2\pi}_0=\dfrac{Q}{4\pi}\cdot 2\cdot 2\pi=Q
\]
2.Calculamos la divergencia de {\bf E} como $\nabla\cdot{\bf E}$:

\[
\nabla\cdot{\bf E}=\dfrac{Q}{4\pi}\left(\dfrac{-2x^2+y^2+z^2}{\sqrt{(x^2+y^2+z^2)^5}}+\dfrac{x^2-2y^2+z^2}{\sqrt{(x^2+y^2+z^2)^5}}+\dfrac{x^2+y^2-2z^2}{\sqrt{(x^2+y^2+z^2)^5}} \right)=0
\]


No se puede aplicar el Teorema de Gauss ya que se requiere que el campo vectorial sea diferenciable. En este caso, el campo no es continuo en el punto $(0,0,0)$, ya que se anulan los denominadores, por lo que no puede ser diferenciable.
\end{document}


  