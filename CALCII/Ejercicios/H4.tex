\documentclass[11pt,a4paper]{article}

\usepackage{latexsym}
\usepackage{amssymb}
\usepackage{amsmath}
\usepackage{amsfonts}
\usepackage{color}
\usepackage{graphicx}
\usepackage{multicol}

\setlength{\textheight}{25cm}
\setlength{\textwidth}{17cm}
\setlength{\topmargin}{-3cm}
\setlength{\oddsidemargin}{-5mm}
\setlength{\evensidemargin}{0.5cm}

\parindent=0mm
\parskip=0mm
\thispagestyle{empty}
\date{22.03.2023}

\newcounter{ex}
\newcommand{\ejer}{\stepcounter{ex}\textbf{\theex.-}\ }
\newcommand{\sen}{\operatorname{sen}}


\begin{document}
\hrule \vskip 1mm

\noindent {\large \sc C\'alculo II.}\\{\large \sc $1^o$ de Grado en Matem\'aticas y Doble Grado Inform\'atica-Matem\'aticas.} \hfill  Curso 2022/23\\{\large \sc Departamento de Matemáticas}


\vskip 1mm

\centering {\bf  Hoja 4 } \vskip 1mm \hrule

\vskip 1mm
\raggedright

\ejer Sea $F(x,y)=f(u(x,y),v(x,y))$ con $u=\dfrac{x-y}{2},v=\dfrac{x+y}{2}$. Aplicar la regla de la cadena para calcular $\nabla F(x,y)$ en función de las derivadas parciales de $f$, $\dfrac{\partial f}{\partial u}$ y $\dfrac{\partial f}{\partial v}$.\\
\vskip 1mm
{\bf SOL:} $\nabla F(x,y)=\left(\dfrac{\partial F}{\partial x},\dfrac{\partial F}{\partial y}\right)=\left(\dfrac{\partial f}{\partial u}\dfrac{\partial u}{\partial x}+\dfrac{\partial f}{\partial v}\dfrac{\partial v}{\partial x}, \dfrac{\partial f}{\partial u}\dfrac{\partial u}{\partial y}+\dfrac{\partial f}{\partial v}\dfrac{\partial v}{\partial y} \right)=\left(\dfrac{1}{2}\dfrac{\partial f}{\partial u}+\dfrac{1}{2}\dfrac{\partial f}{\partial v}, -\dfrac{1}{2}\dfrac{\partial f}{\partial u}+\dfrac{1}{2}\dfrac{\partial f}{\partial v}\right)$\\

\ejer Sean $f(x,y)=x^2+y$, $g(u)=(\sen3u,\cos8u)$ y $h(u)=f(g(u))$. Culcular $\dfrac{dh}{du}$ en $u=0$ tanto de forma directa como usando la regla de la cadena.\\
\vskip 1mm
{\bf SOL:}\\
- De forma directa:
\[h(u)=f(g(u))=f(\sen3u,\cos8u)=(\sen3u)^2+\cos8u\Rightarrow \dfrac{dh}{du}=6\sen3u\cos3u+-8\sen8u\Rightarrow \dfrac{dh}{du}(0)=0\]
- Usando la regla de la cadena:
\[\dfrac{dh}{du}(0)=\dfrac{\partial f}{\partial x}(g(0))\dfrac{d g_1}{d u}(0)+\dfrac{\partial f}{\partial y}(g(0))\dfrac{d g_2}{d u}(0)=\dfrac{\partial f}{\partial x}(0,1)\dfrac{d g_1}{d u}(0)+\dfrac{\partial f}{\partial y}(0,1)\dfrac{d g_2}{d u}(0)=0\]

\ejer Las relaciones $u=f(x,y)$, $x=x(t)$ e $y=y(t)$ definen $u$ como función escalar de $t$, digamos $u=u(t)$. Aplicar la regla de la cadena para la derivada de $u$ respecto de $t$ cuando
\[f(x,y)=e^{xy}\cos xy^2, \hskip 5mm x(t)=\cos t, \hskip 5mm y(t)=\sen t. \]
\vskip 1mm
{\bf SOL:}
\[\dfrac{\partial u}{\partial t}=\dfrac{\partial f}{\partial x}(x(t),y(t))\dfrac{dx}{dt}(t)+\dfrac{\partial f}{\partial y}(x(t),y(t))\dfrac{dy}{dt}(t)=
\]
\[=\left(e^{x(t)y(t)}y(t)\cos(x(t)(y(t))^2)-e^{x(t)y(t)}\sen(x(t)(y(t))^2)(y(t))^2\right)(-\sen t)+
\]
\[+\left(e^{x(t)y(t)}y(t)\cos(x(t)(y(t))^2)-e^{x(t)y(t)}\sen(x(t)(y(t))^2)2x(t)y(t)\right)(\cos t)=   \]
\[=e^{\cos t\sen t}\left( \cos(\cos t\sen^2t)(\cos^2t-\sen^2t)+\sen(\cos t\sen^2t)\sen^3t-2\cos^2t\sen t\right)
\]

\ejer La sustitución $t=g(x,y)$ convierte $F(t)$ en $f(x,y)=F(g(x,y))$. Calcular la matriz $Df(x,y)$ en el caso particular en que $F(t)=e^{\sen t}$ y $g(x,y)=\cos(x^2+y^2)$.\\
\vskip 1mm
{\bf SOL:}\\
Por una parte, $DF(t)=F'(t)$, de forma que $DF(g(x,y)))=e^{\sen(\cos(x^2+y^2))}\cos(\cos(x^2+y^2)$. Por otra, $Dg(x,y)=\left(\dfrac{\partial g}{\partial x}(x,y), \dfrac{\partial f}{\partial y} \right)=(2x\sen(x^2+y^2), 2y\sen(x^2+y^2)$. Luego
\[
Df(x,y)=DF(g(x,y))Dg(x,y)=e^{\sen(\cos(x^2+y^2))}\cos(\cos(x^2+y^2)(2x\sen(x^2+y^2), 2y\sen(x^2+y^2)=
\]
\[
=\left(2xe^{\sen(\cos(x^2+y^2))}\cos(\cos(x^2+y^2)\sen(x^2+y^2), 2ye^{\sen(\cos(x^2+y^2))}\cos(\cos(x^2+y^2)\sen(x^2+y^2)\right)
\]
\end{document}