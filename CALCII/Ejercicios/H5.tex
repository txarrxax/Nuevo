\documentclass[10pt,a4paper]{article}
\usepackage[utf8]{inputenc}
\usepackage{amsmath}
\usepackage{amsfonts}
\usepackage{amssymb}
\usepackage{graphicx}
\usepackage{enumerate}

\setlength{\textheight}{25cm}
\setlength{\textwidth}{17cm}
\setlength{\topmargin}{-2cm}
\setlength{\oddsidemargin}{-5mm}
\setlength{\evensidemargin}{0.5cm}

\parindent=0mm
\parskip=0mm
\thispagestyle{empty}

\newtheorem{thm}{Theorem}

\begin{document}
\textbf{12.-} Calcular los máximos y mínimos absolutos de $f(x,y)=x^3+3xy^2$ en $\Omega=\{(x,y)\in \mathbb R^2:x^2+y^2\leq 1, x\leq y\}$.
\textbf{SOL:}\\
\begin{enumerate}[1.]
\item Calculamos los puntos críticos de $f$ en el interior de $\Omega$.
$$\nabla f(x,y)=(3x^2+3y^2, 6xy)=0 \Rightarrow \begin{cases} 3x^2+3y^2=0\\6xy=0\end{cases}\Rightarrow (x,y)=(0,0)\in \Omega$$
\item Calculamos los puntos críticos de $f$ en la frontera de $\Omega$.\\
\begin{enumerate}[2.1.]
\item Calculamos los puntos críticos de $f$ cuando $x^2+y^2=1, x\leq y$.\\
Definimos $g(x,y)=x^2+y^2-1$ y buscamos los puntos $(x,y)$ tales que $\nabla f(x,y)=\lambda \nabla g(x,y)$ para algún $\lambda \in \mathbb R$.
$$\begin{cases}
3x^2+3y^2=\lambda 2x\\ 6xy=\lambda 2y\\ x^2+y^2=1
\end{cases}$$
Obtenemos los puntos:
$$(x,y)=\left(\pm\dfrac{\sqrt{2}}{2},\pm\dfrac{\sqrt{2}}{2}\right)\in \Omega,
(x,y)=\left(\dfrac{\sqrt{2}}{2},-\dfrac{\sqrt{2}}{2}\right)\not\in \Omega,
(x,y)=\left(-\dfrac{\sqrt{2}}{2},\dfrac{\sqrt{2}}{2}\right)\in \Omega$$
$$(x,y)=(1,0)\not\in\Omega, (x,y)=(-1,0)\in\Omega$$
\item Calculamos los puntos críticos de $f$ cuando $x=y, x^2+y^2\leq 1$.
Definimos $h(x,y)=x-y$ y buscamos los puntos $(x,y)$ tales que $\nabla f(x,y)=\lambda \nabla h(x,y)$ para algún $\lambda \in \mathbb R$.
$$\begin{cases}
3x^2+3y^2=\lambda\\ 6xy=-\lambda\\ x=y
\end{cases}$$
Obtenemos el punto:
$$(x,y)=(0,0)$$
\end{enumerate}
\item Evaluamos $f$ en cada uno de los puntos obtenidos y determinamos los valores máximos y mínimos.
\end{enumerate}
\end{document}